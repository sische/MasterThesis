%\pagestyle{empty}
\begin{abstract}


\noindent Internet of Things (IoT) is known as the concept of connecting everyday physical devices to the Internet. It is natural to assume that the popularity and development within this field will increase in the following years. This means that more and more things will be able to communicate over the Internet. In the process of developing IoT, an important part is to build reliable and scalable networks, and understanding where data should be processed concerning power consumptions and costs of transfering data in different parts of the network.

\noindent The task of the thesis will be to access data in a complete prototype of an IoT network, and both
collect and analyse the data. The goal is to study different alternatives for a typical IoT system, and provide an overview of current state-of-the-art technologies, products and standards that can be usedin such a setting. Data can be generated by using and comparing different sensors connected to end nodes in the network.

\noindent This thesis will therefore aim to test the use of microcontrollers as end nodes in an IoT network. The main focus is to establish a connection between a Raspberry Pi as a central point in a 6LoWPAN network, and connect nRF52 devices from Nordic Semiconductor to this. Using different versions of CoAP to transfer data over a Bluetooth Low energy (BLE) connection, it will be discussed the advantage and disadvantage of sending data rather than doing computation in end nodes, with a main focus on optimal throughput through the network. Optimizing packet sizes, fragmentation and maximising throughput at the same time as minimizing power usage are other key words. 

\noindent To achieve these goals, a central part will be to understand the benefits of processing data in the
end nodes, concerning power, costs and time. This means much less data needs to be sent through
the network. If the calculations needed are too complex, the measured data needs to be transferred
to a central node with higher processing power and easier access of energy. 

\noindent Results from this work includes graphs and discussions in which case the two main versions of CoAP is preferred, and putting them up against each other in the form of tables and graphs from tests done on the IoT system. These show that CoAP NON is grenerally preferable if the data sent is larger than 500 bytes. It is also more stable than CoAP CON in tests performed on this system. However CoAP CON can transfer packets at a higher rate for small packets (< 500 byte), even though each packet needs to receive an ACK, which will use some of the network connection capacity in situations where this is very limited. 



\end{abstract}