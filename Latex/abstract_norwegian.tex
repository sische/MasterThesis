\pagestyle{empty}
\renewcommand{\abstractname}{Sammendrag}
\begin{abstract}
\noindent Tingenes Internet, mer kjent under det engelske navnet Internet of Things (Iot), er konseptet der hverdagslige fysiske gjenstander kobles til Internet. Det er naturlig å anta at populariteten og utviklingen rundt dette vil være økende de kommende årene. Dette betyr at flere og flere ting vil kunne kommunisere over Internet. I prosessen der Tingenes Internet utvikles, er en viktig del å bygge pålitelige og skalerbare nettverk, samt å forstå hvor i nettverket data bør prosesseres med tanke på energibruk og kostnader ved å overføre data mellom deler av nettverket. 

\noindent Hovedoppgaven presentet i denne avhandlingen vil være å jobbe med data i en komplett prototype av et Tingenes Internet-nettverk, og både samle og analysere dataene. Målet er å studere de forskjellige alternativene for et typisk slik nettverk, og lage en oversikt over teknologiene og produktene som kan bli brukt i denne sammenhengen. Data som trengs til dette kan bli hentet ved å sammenligne forskjellige sensorer koblet til løvnodene i nettverket. 

\noindent Med dette som utgangspunkt vil oppgaven bli kul. 


  
\end{abstract}