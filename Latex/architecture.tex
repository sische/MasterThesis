\chapter{System Architecture}
\label{chp:architecture} 

\noindent The purpose of this thesis is to build an end-to-end system, which will be able to transfer data all the way from a \gls{microcontroller} to a server. This chapter will describe in detail how the different components of the testbed are connected, and how the different protocols have been configured to read, process and transfer data efficiently. 

\begin{figure}[ht]
    \centering
    \includegraphics[width=0.89\textwidth]{architectureIntro4.png}    
    \caption{End-to-End architecture in the presented system}
    \label{fig:systemArchitectureThisSystem}
\end{figure}


\noindent Figure \ref{fig:systemArchitectureThisSystem} shows how the complete end-to-end system, the testbed, is set up. The Several \glspl{microcontroller} can be connected to a Pi at the time, forming a star network. Up to eight connections have been tested successfully in the testbed. We will now go through the details of setting up this system. 

\newpage
\noindent There are three main limitations in a system like this:

\begin{itemize}
  \item Computational power in the different nodes
  \item Battery capacity of the end nodes
  \item Network limitations between the nodes
\end{itemize}

%Computations can now either be done in the end nodes at the \textit{nRF52s}, at the \textit{Raspberry Pi}, or forwarded to a web server or another computer with more computational power. Data can also be displayed directly to a web page from the \textit{Raspberry Pi}.

%This means that they are able to communicate using \gls{ipv6} addresses, even though the nRF52 only has a bluetooth antenna built in, as shown in figure \ref{fig:nrf52chipDetail}. The limitations of \gls{ble} means that the nRF52 can only connect to \textit{one} device at the time. Several of these put together are therefore forming a star network using the \textit{Raspberry Pi} as a central point of connection. 


\noindent A central part of the testing in this thesis will be to test the different constraints, and to understand the advantage and disadvantages of doing computations in end-nodes. This will then be compared to transferring  information to a server with higher computational power. Power usage is often closely related to computational power, and will also be a central factor. The next section will contain a walk-through of the testbed, and discuss these three main limitations in each node and the links between them.

%\todo{Move to introduction!}

\section{Connecting Raspberry Pi and nRF52}


%As a microcontroller the nRF52 works good in this network, both as a low-power and powerful device. 
\noindent Since it is not possible to connect a screen to the \gls{nRF52}, it makes sense to connect this to the Pi first, before measuring values. To set up the communication between a \gls{Raspberry Pi} and the \gls{nRF52}, the two code examples TWI and Observable server from Nordic Semiconductor were used as a starting point for coding on the nRF52. Using the Observable server example, it should be possible to observe a field on a server from a client in the testbed. It was however not straightforward to connect these two devices together the first time. A detailed description on how to connect these two can be found in appendix \ref{chp:appendixb1}. 
  

\noindent The nRF52 microcontroller is battery powered using a small \textit{3V Lithium CR 2032} battery. Given this limitation the computational power will be limited as well. A point of discussion will be if it is profitable to handle data here, or if this should be done by more powerful nodes in the network. 

\section{Raspberry Pi to Network Computer or Server} 

\noindent When running the Linux-based \gls{os} Ubuntu Mate, the \gls{Raspberry Pi} can be used more or less like a regular computer. This \gls{os} has a pre-installed version of the most basic programs needed, for instance, \textit{Mozilla Firefox Browser}, \textit{Pluma text editor} and \textit{Linux terminal}. In the testbed, it has been connected to the \gls{lan} using either a wireless or a wired connection. This makes the link from the Pi to another computer very stable and quick, capable of much higher transfer rates than the other links discussed in the system. Since neither the Pi nor the central computer is battery powered in the testbed, this is not an option. Using a forwarding script it is simple to forward data either to the computer or directly to a location on the web, as shown in \ref{fig:systemArchitectureThisSystem}. All these factors implicate that these links will not be a bottleneck in the system. Since this is higher level computer programming and not as limited concerning computational power or battery usage, it will be more interesting to look at the other links in more detail in this thesis. To test these links with higher capacity in more detail will be left to future works, proposed in chapter \ref{chp:results}. 

%\newpage

\section{Connecting nRF52 and ADXL345}

\noindent The used \gls{ADXL345} was connected using \gls{i2c}, which is supported by the \gls{nRF52}. Connection scheme is as follows (\gls{nRF52} $\,\to\,$ \gls{ADXL345}): 

\begin{table}[H]
\centering
\caption{Connection scheme nRF52 to ADXL345}
\label{nRF52ADXL345connection}
\begin{tabular}{ll}
\textbf{Pin connection} & \textbf{Explanation}                                                                                            \\
5V -- VIN               & Power source, \textbf{\textcolor{green}{green cable}} in Figure \ref{fig:nrf-adxl345}                   \\
GND -- GND              & Ground, \textbf{\textcolor{red}{red cable}} in Figure \ref{fig:nrf-adxl345}                             \\
P0.27 -- SDA            & \gls{i2c} Serial Data Line, \textbf{\textcolor{orange}{orange cable}} in Figure \ref{fig:nrf-adxl345} \\
P0.26 -- SCL            & \gls{i2c} Serial Clock Line, \textbf{\textcolor{brown}{brown cable}} in Figure \ref{fig:nrf-adxl345} 
\end{tabular}
\end{table}

\noindent \gls{nRF52} supports both \gls{i2c} and \gls{spi} serial computer buses. \gls{i2c} was chosen in this case because it is fast enough, flexible and simple to set up with the use of few cables. As seen in table \ref{nRF52ADXL345connection} and figure \ref{fig:nrf-adxl345}, this interface only requires four cables, for power, ground, data and clock. This gives a bandwidth of 1 bit and a maximum bitrate of 5 Mbit/s \cite{semiconductors2000i2c}. 

\begin{figure}[ht]
    \centering
    \includegraphics[width=0.62\textwidth]{connectionADXL-nrf5.png}    
    \caption{Connected nRF52 -- ADXL345}
    \label{fig:nrf-adxl345}
\end{figure}

\newpage


\noindent After the physical connection was complete, it was possible to start the process of initializing the \gls{ADXL345}. Acceleration values can only be read from this sensor if this has been correctly initialized at compile time. To do so, code to write to and read from the registers had to be added. We used another example from Nordic Semiconductor as a starting point to establish this communication. This was called \textit{TWI master with TWI slave}. By using specific methods from this example and writing to the right accelerometer registers in the right order \cite{devices2009digital}, it was possible to configure the accelerometer as wanted. The detailed description of the programming code used to do this can be seen in appendix \ref{subsec:progInC}. 

\noindent It turned out to be difficult and very time consuming to configure the \gls{ADXL345} to work as expected with the \gls{nRF52}. In short, we found that we were able to measure 11 times for every main loop of the code, and 150 within these 11 loops. This resulted in 1650 measurements every second. The \gls{ADXL345} updates its acceleration value when instructed by the master, and the default setting is to follow the oscillator \textit{tick} of the \gls{nRF52}. This gives an update approximately every second.  The result was that even though the register was being read as often as possible, the same value was read up to 1650 times before it was updated.

\noindent To solve this problem the default setting of updating the register when told by the oscillator needed to be changed. This turned out to be very time-consuming and hard to solve in a proper way. Both because of problems with initializing the accelerometer correctly and making the \gls{nRF52} read and store the values fast enough to get proper data. The ideal solution would be to read at least 1000 values every second, to get a good starting point before analysing values. At this point it was not possible to get enough real data to be used in data analysis in another device in the testbed. In order not to loose too much time on hardware problems, it was decided to focus more on analysing the data sent over the network communication with random generated data. 

%\noindent It would have been possible to use these values to test the network, even though the same value were read several times in a row. However, the accelerometer did never give a fixed number of measured values, and all the measured values could be from 0 to 5 digits long. Since the main focus of this thesis is to understand how data travels through the network, it was decided to not focus further on getting real data due to hardware problems. It would be preferable for the network measurements to use predictable data of fixed length, to get directly comparable results. To test the system with real data will therefore be discussed in the future work section, in chapter \ref{chp:results}. Since the network connection between the \gls{nRF52} and \gls{Raspberry Pi} was already stable, it was easy to generate random values of fixed length to send on the \gls{nRF52}, and do measurements to calculate the optimal throughput between these two. 

%\noindent The next chapter will describe the data analysis of the data sent through this network in detail, and how to optimize the percentage of usable data being transported.  

%\newpage

\section{Discussion}

\noindent Now the full system shown in figure \ref{fig:systemArchitectureThisSystem} has been connected. Due to problems explained in the previous section, the rest of the thesis will focus mainly on the link between the \gls{nRF52} and the \gls{Raspberry Pi}, with the option of using extra computational power from the stationary computer or a web service if necessary. The central point of discussion at this point is how to process and analyse data in the system. The main options to consider in all the different devices concerning how much computation to do on the \gls{Raspberry Pi} are the following: 

\begin{itemize}
  \item Useful raw data: All data arrives as useful data, and can be posted directly to a web page or a server for storage
  \item No computation: Forward all data directly to a computer with more computational power
  \item Some computation: Analyse the data to find data that is not relevant to filter out
  \item Full computation: Do a full analysis of the data. The results can then be posted directly to a server or displayed on a web page. 
\end{itemize}

%\todo{Forenkle. Se Franks kommentarer.}
\noindent The most relevant option of these four depends on the data, and on how much computational power is needed. It is possible to run the \gls{Raspberry Pi} from a power bank, but this has not been tested in this project. When set-up without a battery as power source, the Pi is the first node that could do computations without having to take power limitations as a major concern. The main limitation here is computational power, while the main limitation may be battery power on the \gls{nRF52}. Therefore, it makes sense to do some easy computation on this device. For instance, if this network is being used to measure vibrations, it is reasonable to assume that measurements more frequent than once every 100 \textit{ms} is needed. Any less frequent than this and vibrations could be missed, especially if it is periodical. It would then be perfectly reasonable to assume that the Pi could go through these values, and calculate whether or not the current acceleration value has exceeded a given threshold. This result can then be displayed directly on a web page or a connected monitor from the Pi. If however the system is to calculate \textit{patterns} in the acceleration values, several values need to be compared together. The need of complex algorithms to find these patterns is expected before the results can be displayed. In this case, it is reasonable to assume that the Pi would need additional computational power. The Pi can then be set up as a forwarding device, that forward data directly to a computer with more computational power. 



%\subsection{Connection Challenges} %\todo{Move to appendix, or change? Typ: Planned to this, that didnt work because of this, we did this instead}

%\noindent Following instructions from a representative from Nordic Semiconductor, the bt0 interface was set up using the recommended \gls{ipv6} prefix: \textit{2001:bt8::1/64} in in the file \textit{/etc/sysctrl.conf}, and after that trying to connect and test the connected devices using addresses on the form: 

%\begin{verbatim}
%ping6 2001:bt8::0211:22FF:FE33:4455
%\end{verbatim}

%\noindent This turned out not to work on the \gls{ipv6} network used on \gls{ntnu}, where the standard prefix is \textit{2001::1/64}, without the \textit{db8}. Both solutions should however work in other \gls{ipv6} networks. 



