\chapter{Appendix C}
\label{chp:appendixb1}

Appendix \ref{chp:appendixb1} contains detailed description on how to connect and configure the different devices in the testbed. 

\section{Connecting Raspberry Pi and nRF52}

\noindent Following is a listing of Linux terminal commands for the \gls{Raspberry Pi}, to get the testbed up and running \cite{nordicNrfDocumentation}. 

\noindent Install an \gls{os} on the Raspberry Pi that has a Linux kernel version later than 3.18. On \textit{Raspbian} version 3.18 is the only stable version, (Note: Jan. 2016), but \textit{Ubuntu Mate} is stable in version 4.15. Ubuntu Mate was therefore chosen as the best and most stable \gls{os}, and was installed on the memory card from another computer \cite{ubuntuMate}. When this is done, a resizing of the file system is needed to use all the capacity of the memory card. This is not crucial to get the \gls{os} up and running, but recommended to be able to use more than 4GB of the memory card. Recommended size of the memory card is 16GB. To resize, after the initial boot of the \gls{os} on the \gls{Raspberry Pi}, run the following commands: 

\begin{verbatim}
sudo fdisk /dev/mmcblk0
\end{verbatim}

\noindent Delete partition (d,2), and run the following after a reboot

\begin{verbatim}
sudo resize2fs /dev/mmcblk0p2
\end{verbatim}

\noindent All the following commands require admin rights on the system. It is therefore easier to type in the following command to temporarily become a \textit{super user}. Alternatively type in \textit{sudo} before every command in the rest of the recipe.

\begin{verbatim}
sudo su
\end{verbatim} 

\noindent It should now be possible to exploit the whole memory card, and start downloading and activating services needed in the system. To use \gls{ble}, install Bluez and radvd using \textit{apt-get}:

\begin{verbatim}
apt-get install radvd
apt-get install bluez
apt-get upgrade
apt-get update
\end{verbatim}
%\end{lstlisting}

\noindent \gls{ipv6} forwarding is needed to let the end nodes discover each other through the central node in the star network. To activate this, uncomment the following line (remove "\#") in the file \textit{/etc/sysctl.conf}

\begin{verbatim}
net.ipv6.conf.all.forwarding=1
\end{verbatim}

\noindent To find the \gls{ipv6} prefix in the network, run the command \textit{ifconfig}. Find a field named \textit{inet6 addr}, and write down the first and last number on this line (For instance 2001 and /64). 
The communication will in this case go through a custom designed interface. This will be named bt0. Start by creating the \textit{radvd.conf}-file, and open it for editing. 

\begin{verbatim}
touch /etc/radvd.conf
pico /etc/radvd.conf
\end{verbatim} 


\noindent Write in the following bt0 interface. Replace the number 2001 and /64 with the numbers found in the previous step. 

\begin{verbatim}
interface bt0
{
    AdvSendAdvert on;
    prefix 2001::/64
    {
        AdvOnLink off;
        AdvAutonomous on;
        AdvRouterAddr on;
    };
};
\end{verbatim} 

\noindent To mount the modules \textit{bluetooth\_6lowpan, 6lowpan and radvd}, add the following to \textit{/etc/modules}. If the file does not exist, create it by entering \textit{touch /etc/modules} first. 

\begin{verbatim}
bluetooth_6lowpan
6lowpan
radvd
\end{verbatim}

\noindent When the system is booted, these modules will be automatically loaded. The \textit{hcitool} command should now be available. This is a tool designed to connect and keep track of connected devices, both through standard bluetooth and \gls{ble}. 

\begin{verbatim}
hcitool lescan
\end{verbatim}

\noindent \textit{lescan} will scan for \gls{ble} devices nearby, and find the bluetooth address, for instance \textit{00:AA:11:BB:22:CC}. The normal procedure in this case would be to run the following command: 

\begin{verbatim}
echo 1 > /sys/kernel/debug/bluetooth/6lowpan_enable
hcitool lecc 00:AA:11:BB:22:CC
service radvd restart
\end{verbatim}

\noindent These commands never established a stable connection in this system. It was not possible to test the connection, and each connected device became automatically disconnected after about 15 seconds. We never found the reason for this problem. Instead, it was possible to not use \textit{hcitool} for this part. The following commands worked fine:

\begin{verbatim}
cd /sys/kernel/debug/bluetooth
echo 1 > 6lowpan_enable
echo "connect 00:AA:11:BB:22:CC 1" > 6lowpan_control
service radvd restart
\end{verbatim} 

\noindent The command \textit{hcitool con} shows the connected \gls{ble} devices. If the device is connected, the connection can be tested by typing:

\begin{verbatim}
ping6 2001::02AA:11FF:FEBB:22CC
\end{verbatim}


\noindent Note that \textit{2001::02AA:11FF:FEBB:22CC} is the full \gls{ipv6} address of the device when the Bluetooth address is \textit{00:AA:11:BB:22:CC} in the testbed. The \gls{ipv6} address can be used to route packets using \gls{6lowpan}. Using the basic examples provided by Nordic Semiconductor described in chapter \ref{chp:architecture}.1, it was now possible to send messages both using \gls{coap} \gls{con} and \gls{non}.
%\todo{3.3, still right reference?}


\section{Connecting nRF52 and ADXL345}

In short, registers for \textit{data format control, initial power saving, interrupt enable control}, and \textit{the offset of each axis} has to be written to in that order. After this, the acceleration value from the different axes can be read. It was then possible to read from the registers containing current acceleration values using the method \textit{read\_reg} described in the next section.  

\noindent In the solution proposed in this thesis, the acceleration values are being read as often as possible, limited by the processing power of the \gls{nRF52} and the \gls{i2c} connection. Furthermore, the read value is being stored in a simple dynamic char array in the \gls{nRF52} before being sent and reset when the \gls{ble} channel is ready. The highest obtained measurement frequency in this system was 11 times for every main loop, and 150 within these 11 loops. This resulted in 1650 measurements every second, but as explained in chapter \ref{chp:architecture}, even though the register was being read as often as possible, the same value was read up to 1650 times before it was updated. The accelerometer was therefore not tester further in this project. 

\noindent The next section contains samples of programming code written to read acceleration data from the Adafruit gls{adxl345} connected to the \gls{nRF52} using the \gls{i2c} interface. This code was not being used in the testing of this thesis, as explained in chapter \ref{chp:architecture}. The code has been included and explained so it can be used by others in later projects.

\section{C programming code for acceleration data} \label{subsec:progInC}

The following code sample in C programming is parts of the main function in the file \textit{main.c}. From here methods \textit{accelerometer\_init} and \textit{start\_measuring} are being called to intitialize the different registers of the accelerometer, and start the measuring from the main loop. 
\newpage
\begin{lstlisting}[language=C]

int main(void){
	uint32_t err_code; 
	
	app_trace_init(); 
	leds_init(); 
	timers_init();
	accelerometer_init(); 
	
	...
	
	for (;;)
	{
		power_manage();
		start_measuring();
	}
}
\end{lstlisting}

%\newpage

\textit{accelerometer\_init} will initialize the different registers to be able to read from the accelerometer. These registers should first be defined in a header file along with information about the slave address and which \gls{nRF52} pins that represented SCL and SDA, to clarify the code. 

\begin{lstlisting}[language=C]
// Part of header file: 

	#define ADXL345_SLAVE_ADDRESS			0x53
		
    #define TWI_SCL_M						27   //!< Master SCL pin
    #define TWI_SDA_M						26   //!< Master SDA pin
		
	#define X_AXIS_OFFSET					0x1E
	#define Y_AXIS_OFFSET					0x1F
	#define Z_AXIS_OFFSET					0x20
	#define DATA_RATE_AND_POWER_INIT		0x2C
	#define POWER_SAVING_INIT				0x2D
	#define INTERRUPT_ENABLE_CONTROL		0x2E
	#define DATA_FORMAT_CONTROL				0x31
	#define READ_X_AXIS						0x32
	#define READ_Y_AXIS						0x34
	#define READ_Z_AXIS						0x36
\end{lstlisting}
\newpage
\begin{lstlisting}[language=C]

// Initialize accelerometer in main.c file: 

static void accelerometer_init()
{
	write_reg(DATA_FORMAT_CONTROL, 0x00, 2);
	write_reg(POWER_SAVING_INIT, 0xFF, 2);
	write_reg(INTERRUPT_ENABLE_CONTROL, 0xFF, 2);
	write_reg(X_AXIS_OFFSET, 0xFF, 2);
	write_reg(Y_AXIS_OFFSET, 0xFF, 2);
	write_reg(Z_AXIS_OFFSET, 0xFF, 2);	
}

\end{lstlisting}

The initialization of the accelerometer calls the function \textit{write\_reg}, which is used to write to a register. 

\begin{lstlisting}[language=C]
static uint32_t write_reg(uint8_t register_address, uint8_t data_to_write, uint8_t size) 
{		
	ret_code_t ret;
	
    uint8_t addr8 = (uint8_t)register_address;
    ret = nrf_drv_twi_tx(&m_twi_master, ADXL345_SLAVE_ADDRESS, &addr8, 1, true);
    if(NRF_SUCCESS != ret)
    {
        break;
    }
    ret = nrf_drv_twi_tx(&m_twi_master, ADXL345_SLAVE_ADDRESS, &data_to_write, size, false);
    return ret;
}

\end{lstlisting}

After this initialisation process is successful, the \textit{start\_measuring} can begin. 

\begin{lstlisting}[language=C]

static void start_measuring()
{
	char stringa[150];
	char anotherString[150];
	
	for (int j = 0; j < 150; j++)
	{	
		int r = read_reg(READ_Z_AXIS, 0x00);
		int t = numberOfMeasurements++;
	}
	
	sprintf(stringa, "%d,", numberOfMeasurements);
	
	for  (in i=0; i < 200; i++)
	{
		if (stringa[0] == '\0')
		{
			measuringCounter = i;
			break;
		}
		else
		{
			if (!stringToSendOccupied)
			{
				appendChar(stringToSend, 150, stringa[i]);
			}
		}
	}
	numberOfMeasurements = 0; 
}

\end{lstlisting}

This method calls the function \textit{read\_reg}, to read the registers that have been set to update earlier. 

\begin{lstlisting}[language=C]

static uint16_t read_reg(uint8_t register_address, uint8_t data_returnValue) 
{
	uint16_t rd;
	ret_code_t ret;
	uint8_t buff[2];
    uint8_t addr8 = (uint8_t)register_address;
    
    ret = nrf_drv_twi_tx(&m_twi_master, ADXL345_SLAVE_ADDRESS, &addr8, 1, true);
    if(NRF_SUCCESS != ret)
    {
        break;
    }
    ret = nrf_drv_twi_rx(&m_twi_master, ADXL345_SLAVE_ADDRESS, buff, 2, false);
	rd = (uint16_t)(buff[0] | (buff[1] << 8));
    
    return rd;	
}

\end{lstlisting}

After this it is possible to get the acceleration value from another point in the code, to be stores in a char array \textit{*str} until it is being sent. 


\begin{lstlisting}[language=C]

static void acceleration_value_get(coap_content_type_t content_type, char ** str)
{
	stringToSendOccupied = true; 
	
	strcpy(newString, stringToSend);
	*str = newString;
	stringToSend[0] = '\0';
	
	stringToSendOccupied = false; 
}


\end{lstlisting}

